\documentclass[10pt]{article}
\usepackage[utf8]{inputenc}
\usepackage[papersize={11in, 17in}]{geometry}
\usepackage[absolute]{textpos}
\TPGrid[0.5in, 0.25in]{23}{24}
\usepackage{palatino}
\parindent=0pt
\parskip=12pt
\usepackage{nopageno}
\begin{document}

\begin{textblock}{23}(0, 1)
\center \huge PREFACE
\end{textblock}

\begin{textblock}{23}(0, 2.5)

\textbf{Al-kitab al-khamr} is the book of forbidden drink. ``Khamr" is the word
in the Q\={u}r'an that prohibits the faithful from intoxicants: from wine and
from stimulants and from bringers-of-visions. What things must those be that
between poison and pleasure tack course in the body? What colors and shapes the
forbidden inscribes as its left-behind marks on dreams and insoluble mind.

\textbf{Instrumentation:}

\begin{itemize} \itemsep2pt
\item Bass flute (doubling flute)
\item English horn (doubling oboe)
\item Bass clarinet (doubling B$\flat$ clarinet)
\item Baritone saxophone (doubling sopranino saxophone)
\item Guitar
\item Piano
\item Percussion
\item Violin
\item Viola
\item Cello
\item Contrabass
\end{itemize}


\textbf{Prioritization of tempo.} The piece comprises two series of different
tempi. Tempo series one sets the quarter note equal to 126, 63 or 31.5 (written
as 32). Tempo series two sets the quarter note equal to 84 or 42. The tempi of
the first series stand 3:2 in relation to the tempi of the second series.
Even though the choice of tempi are to some extent a matter of the preferences
of the ensemble and the acoustics of the hall, the subito changes of tempo in
the piece should be felt and conducted as exact metric modulations. In
addition, the tempi of the very fast parts of the piece should be played as
closely as possible to the tempi written in the score: it is preferable to play
the dense figures in very fast parts of the piece as something of a blur rather
than slowing the tempi to attack the notes carefully.

\textbf{Stopping time.} Fermatas are not (yet) written in the score. But
fermatas should be inserted by the conductor in the places that need them. All
the measures written as grand pauses are fair game for fermatas. As are
individual beats that help clarify the intensity of transitions from one type
of material to the next.

\textbf{Accidentals.} Accidentals govern only one note. This is true even for
successive noteheads at the same staff position. \textit{Because of this no
natural signs appear in the score} (with the exception of parenthesized
noteheads in trills). The sequence of, for example, G$\sharp$4 followed by
G4 (without accidental) is to be understood as G$\sharp$4 followed by
G$\natural$4.

\textbf{Barlines.} Four barlines are missing in the score (immediately prior to
each of the four rehearsal marks). The missing barlines mean nothing and will
be included in a later version of the score.

\textbf{The winds are tranposed.} The bass flute sounds an octave lower than
written. The English horn sounds a perfect fifth lower than written. The
B$\flat$ clarinet sounds a major second lower than written and the bass
clarinet sounds a major ninth lower than written. The baritone saxophone sounds
a major thirteenth lower than written and the sopranino saxophone sounds a
minor third higher than written.

\textbf{Flute.} The two bass flute multiphonics in the piece are numbers 17 and
22 in Carin Levine's book \textit{Die Technik der Flötenspiel} and the boxed
numbers in the score are reminders of this. Any fingerings approximating the
off-octave sound of the multiphonics may be used. Trills without secondary
noteheads are color trills.

\textbf{Saxophone.} The multiphonic dyad in the piece is number 77 in Marcus
Weiss's book \textit{Die Technik der Saxophons}; the boxed number in the score
is a reminder of this.

\textbf{Guitar.} The guitar is tuned as usual. The sound ideal for all plucked
notes is as resonant as possible; interpret rests only as rhythmic placeholders
(and not as indications to stop the reverberation of the notes). Cross
noteheads indicate half harmonics; play the low E (or other open strings)
marked this way with a type of RH plucking that best approximates the color of
the other half harmonics. Individuated clicks indicated in the score should be
executed by running a pick or fingernail laterally up the outer wire weave of
the E string creating a continuous but sparse and irregular sound. Use a metal
machinists screw of about 8 or 10 centimeters like a type of corrugated guiro
in the part of the score that requests screw-bowing; make up-bow and down-bow
changes freely.

\textbf{Piano.} The piano should be prepared with a piece of cardboard woven
between the strings of twelve notes in the octave from F$\sharp$6 to
F$\sharp$7. The effect is coarsely to mute these pitches; no special indication
is given in the score when these pitches are encountered. `Tamburo' hits
characterize the first section of the piece. Execute these with heel of the
palm struck against the lowest strings inside the piano with the sustain pedal
lifted; the sound augments the color of the tam-tam. Use a credit card run very
slowly laterally up the weaving of the low C$\sharp$1 string in the part of the
score that requests individuated clicks.

\textbf{Percussion.} Six percussion instruments are required: (1.) one
woodblock; (2.) mounted castanets; (3.) snare drum; (4.) bass drum; (5.) very
large tam-tam (38" recommended); (6.) marimba. The percussion part is notated
primarily on a single-line staff. Where cells of the five-line staff occur they
represent a synchoronous attack on A$\flat$5 in the marimba together with a
single woodblock; these two instruments are always struck together in the piece
and should be placed near each other so that each can be hit with a hard mallet
at the same time. The tam-tam should be as large as possible and the tam-tam
dynamics written into the score may be freely ignored: the goal is as resonant
a sound that fills as much of the hall as possible without spilling over from
the fundamental of the instrument into the less desirable upper frequencies.
Rolls on the bass drum are all to be as close to attackless as possible: the
rate of the roll doesn't matter but the background depth provided by the
instrument is important.

\textbf{Strings.} The violin, viola and cello are tuned as usual. String IV of
the contrabass is tuned down to G$\natural$0 (a major sixth lower than the
usual tuning of E$\natural$1) and will probably be a little loose as a result.
(Note that that the seemingly large double stops in the contrabass at the
interval of a minor seventh are all played with the fingers at the exact same
position on strings III and IV.) The contrabass plays a special role in the
piece and should be allowed to sound front-and-center above the other strings
in many sections of the piece. Natural harmonic glissandi lentissimi in the
violin, viola, cello and contrabass are designed to encourage the production of
multiphonics and other unstable harmonics: allow the multiphonics and transient
harmonics to sound as much as possible and do not adjust them back to
recognizable harmonics unnecessarily.

\end{textblock}

\begin{textblock}{23}(0, 23)

\textbf{Al-kitab al-khamr} was written between January and April 2015 for
Ensemble Dal Niente. The piece is to be premiered on the 16\textsuperscript{th}
of May 2015 on the campus of Harvard University by Ensemble Dal Niente.

\end{textblock}

\end{document}